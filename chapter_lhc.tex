\chapter{LHC and the ATLAS detector }\label{chap:lhc}

[Thesis 2011-258]

This chapter  takes a closer look at the Large Hadron Collider located at CERN,
the European Organization for Nuclear Research, and at one of the detectors placed
along its ring: ATLAS, A Toroidal Lhc ApparatuS (described in detail in Refer-
ence [ATLAS Collaboration 2008a]).


The Large Hadron Collider (LHC) \cite{lhc1}, \cite{lhc2} is a proton-proton collider in Geneva,
Switzerland that collides two beams of protons together at very high energies. The ATLAS
detector, which is designed to measure the output of these collisions, is located at one of
four collision points on the LHC accelerator ring. During 2012, the third year of operation,
the proton-proton collisions have been produced at a center-of-mass energy of $\sqrt{s}= 8$ TeV
with 4 TeV per proton. This center-of-mass energy has been chosen to ensure a safe
operating margin for the magnets in the accelerator, avoiding damage due to resistive
connections [85]. The design center-of-mass energy of the LHC is $\sqrt{s}= 14$.


\section{The Large Hadron Collider at CERN}
The Large Hadron Collider (LHC) is a circular accelerator located at CERN and designed
to collide beams consisting of protons or heavy-ions. It is currently the highest energy
collider in the world since it collided proton-beams at a center of mass energy of $\sqrt{s}= 7$

\subsection{Accelerator complex}

The accelerator complex at CERN is an ensemble of machines capable of accelerate particles
at increasingly higher energies. Each machine injects the beam into the next one, which takes over to bring the beam to a higher energy, and so on. The accelerator complex
is schematically shown in Figure \ref{accl}. The very first step in the chain is the proton source. The protons, extracted from Hydrogen gas, are fed into a linear accelerator (LINAC2). The LINAC2 accelerates the protons to an energy of 50 MeV. At the end of the LINAC2, the protons are injected in the Proton Synchrotron Booster (PSB), a circular accelerator in which the protons reach an energy of 1.4 GeV. At this energy, the protons are ready to be injected in a second circular accelerator,
the Proton Synchrotron (PS), in which they are accelerated up to 25 GeV. After the PS, the
protons are injected in a third circular accelerator, the Super Proton Synchrotron (SPS), in
which their energy arises to 450 GeV, which is the injection energy for the Large Hadron
Collider. The LHC is the last and more powerful step of acceleration, which boosts the proton to
4 TeV durint 2012. It is located in a circular tunnel 27 km km in circumference. The tunnel is buried about
50 to 175 meters underground. It straddles the Swiss and French borders on the outskirts
of Geneva.

\begin{figure}[h]

\includegraphics[width=\textwidth]{figs/accelerator}
\caption{
LHC accelerator complex
}
\label{accl}

\end{figure}



The beams move around the LHC ring inside a continuous vacuum guided by magnets.
The magnets are superconducting and are cooled by a cryogenics system, which makes
the LHC, not only the highest-energy collider in the world, but also the largest cryogenic
system.
The accelerator is made of eight arcs and eight "insertions". Each arc contains 154
dipole magnets. An insertion consists of a long straight section plus two (one at each end)
transition regions. The exact layout of the straight section depends on the specific use of
the insertion: physics (beam collisions within an experiment), injection, beam dumping,
beam cleaning. The important parameters that characterize the LHC with the designed
values and the values reached at the end of 2010 are listed in Table 1.1.
Once the proton bunches are injected and accelerated, the beams are stored at high
energy for hours. During this time collisions take place in the interaction points inside the
four main LHC experiments.
Figure





\subsection{Luminosity}

\section{The ATLAS detector}
\subsection{Overview}
%\subsection{ATLAS Coordinate system}
\subsection{Inner Detector}
\subsection{Calorimeter}
\subsection{Muon Spectrometer}
\subsection{Magnet system}
\subsection{Trigger and Data acquisition}
\subsection{Detector Simulation}
%\subsection{2012 Monte Carlo samples}
%\subsection{2012 data}


