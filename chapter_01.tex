
\chapter{INTRODUCTION}\label{chap1:introduction}

{\bf Put your introduction here.}


Typesetting in \LaTeX \ is {\em the} way to communicate mathematics
(journal articles, books, MS theses, doctoral dissertations),
and \LaTeX \ is also popular in other disciplines, such as 
physics or computer science.
This template should make the formatting of your 
thesis easier. It is set up so that, unless
there are some really wide equations or figures, 
all margins should automatically be correct.
Similarly, all parts are in the order required by the graduate school and 
the required tables (listing contents, figures, tables)
are created automatically in the required format. 
The template is being compiled by running \LaTeX \ on 
the file phd\underline{~}thesis.tex.
The only part of 
phd\underline{~}thesis.tex
that you should change are the 
$\setminus $include$\{ \ldots \} $
commands:
Add more to accommodate all the parts of your thesis
and comment out those that do not apply. 


For the actual software, Mik\TeX \ (see \cite{MiKTeX})
is a standard \LaTeX \ compiler for PCs and
WinEdt (see \cite{WinEdt}) is a standard front end.
WinEdt asks for a registration fee. 
The similar \TeX nicCenter (see \cite{TeXnicCenter})
is free. 


